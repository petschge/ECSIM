\documentclass[%
%reprint,
%superscriptaddress,
%groupedaddress,
%unsortedaddress,
%runinaddress,
%frontmatterverbose, 
preprint,
%preprintnumbers,
%nofootinbib,
%nobibnotes,
%bibnotes,
amsmath,amssymb,
aps,
%pra,
%prb,
%rmp,
%prstab,
%prstper,
%floatfix,
]{revtex4-2}

\usepackage{subcaption} % for subfigures
\usepackage{graphicx}% Include figure files
\usepackage{dcolumn}% Align table columns on decimal point
\usepackage{bm}% bold math
\usepackage{placeins} % FloatBarrier
\usepackage{xcolor} % For remarks between ourselves
\usepackage{listings} % For source code and output

%\usepackage{hyperref}% add hypertext capabilities
%\usepackage[mathlines]{lineno}% Enable numbering of text and display math
%\linenumbers\relax % Commence numbering lines

%\usepackage[showframe,%Uncomment any one of the following lines to test 
%%scale=0.7, marginratio={1:1, 2:3}, ignoreall,% default settings
%%text={7in,10in},centering,
%%margin=1.5in,
%%total={6.5in,8.75in}, top=1.2in, left=0.9in, includefoot,
%%height=10in,a5paper,hmargin={3cm,0.8in},
%]{geometry}

\begin{document}

%\preprint{APS/123-QED}

\title{Note on implementation of a semi-implicit, energy-conserving PIC}% Force line breaks with \\


\author{Patrick Kilian}
\affiliation{
Space Science Institute
}%


\date{\today}% It is always \today, today,
             %  but any date may be explicitly specified

%\begin{abstract}
%\end{abstract}

%\keywords{Suggested keywords}%Use showkeys class option if keyword
                              %display desired
\maketitle

\section{Introduction}

This PiC code follow closely the reference in \cite{LAPENTA2017349}. The basic
idea is to keep particle updates in position and velocity full explicit, but to
include changes to the current from future fields. This leads to the
requirements for deposition of mass matrices and for solving a matrix equation
for the future fields, but no subcycling is necessary. The method conserves
energy and can be used on cell sizes much larger than a Debye length since the
Finite-Grid-Instability is removed.

\section{Validation Tests}
\subsection{Two-Stream Instability}

This test follows the benchmark test described in \cite[section
4.1]{LAPENTA2017349}.  Pick a reference density $n_0$ for normalization.
Compute plasma frequency $\omega_{pe} = \sqrt{4\pi\,n_o\,e^2/m_e}$ and skin
depth $d_e = c/\omega_{pe}$. The simulation domain is only resolved in one
spatial dimension and is $L_x = 2\pi\,d_e$ long and resolved by $N_x = 64$
cells. This implies $\Delta{x} = L_x/ N_x = \pi/32 d_e \approx .098 d_e$, or in
terms of Debye length $\Delta{x} \approx 9.8 \lambda_D$. This resolution is
borderline, but feasible with regular PiC codes.  We use a timestep of
$\Delta{t} = 1/8 \omega_{pe}^{-1}$ and follow the system for $T = 125
\omega_{pe}^{-1}$, which implies we perform $N_t = 1000$ time steps. The CFL
condition number is given by $\mathcal{C} = c \Delta{t}/\Delta{x} = 1/8 c
\omega_{pe}^{-1} / (\pi/32 c \omega_{pe}^{-1}) = 4/\pi \approx 1.27$. This is
again borderline, it's possible in electrostatic explicit PiC codes
($\mathcal{C} < \pi/2$), but not in (most) electromagnetic PiC codes.  Two
electron populations of equal density $n_0/2$ are initialized with beam speeds
$v_{beam,x} = \pm 0.2 c$, thermal spread $v_{th,e} = 0.01 c$ and a small
perturbation $\delta{v} = 0.001 c$.

We use exactly 154 particles per cell (following \cite{LAPENTA2017349}, even
though 156 ppc would get us closer to $10^4$ particles in total) positioned at
offsets that are distributed uniformly randomly in the cell for each particle
in that cell.  Linear theory predicts (according to \cite{LAPENTA2017349}),
that the fastest growing mode has $m = 3$ (corresponding to $k = 2\pi\,m/L_x$)
and its growth rate is $\gamma = 0.35 \omega_{pe}$. And this is also the $m$
that we pick for the small perturbations of particle velocities.  We do not set
any background magnetic fields.

Since this is an one dimensional electrostatic problem we can run this test
case with all three different simulation codes:
\begin{enumerate}
    \item 1d1v phase space, only solving for $E_x$. $B_x$ is required to be
    homogeneous due to the divergence constraint. It will remain unchanged in
    time since there is no currents $j_{y,z}$ and it would only modify
    $v_{y,z}$ and can hence be completely removed from the code.
    \item 1d1v phase space, but retaining $E_x$ and $B_x$. $B_x$ of course
    still needs to remain homogeneous and static, but we can check if the
    Maxwell solver (the most complicated part of the algorithm) reproduces it,
    instead of enforcing it be construction.
    \item 1d3v phase space, retaining all six field components. Thermal
    fluctuations in $v_{y,z}$ will produce transverse current and hence
    transverse fields, but none of these fluctuations should have a strong
    impact on the electrostatic instability. Late term behaviour might be
    changed due to particles scattering of magnetic field fluctuations.
\end{enumerate}

To check the validity of the code we look at three diagnostics:

\begin{figure}
    \centering
    \includegraphics[width=0.55\textwidth]{../tests/twostream/figure1_comp.pdf}
    \caption{Growth of the electric energy in the three simulation, compared
    against the analytic prediction of $\exp\left(2\gamma t\right)$. Compare
    with \cite[Figure 1]{LAPENTA2017349}.}
    \label{fig:twostream_fig1}
\end{figure}

In figure \ref{fig:twostream_fig1} we reproduce the first result figure of this
plot in the reference paper and look at the growth of the electric field energy
$\mathcal{E}_E = \sum_g (E_x^2 + E_y^2 + E_z^2) V_d / (8 \pi)$ as a function of
time and compare against the analytic growth rate. Both 1d1v runs produce
identical results to withing machine precision and are not distinguishable in
the plot. The 1d3v run agrees on growth rate and saturation but shows somewhat
different late time behaviour.

\begin{figure}
    \centering
    \includegraphics[width=0.55\textwidth]{../tests/twostream/figure2.pdf}
    \caption{Change in total energy over the full simulation duration for all
    three simulations. Compare with \cite[Figure 2]{LAPENTA2017349}.}
    \label{fig:twostream_fig2}
\end{figure}

In figure \ref{fig:twostream_fig2} we reproduce the second result figure and
check conservation of total energy. With carefully selected tolerances in the
matrix solver we achieve $\left|E(t) - E_0\right| < 10^{-15} E_0$ for the
duration of the simulation run. This is using a setting of \texttt{tol} =
$10^{-6}/c$ and \texttt{atol} = $10^{-15}$ for nearly all timesteps,
automatically falling back to \texttt{tol} = $10^{-5}/c$ or \texttt{tol} =
$10^{-4}/c$ where GMRES fails to converge with this setting. Larger values for
\texttt{tol} are possible, but relaxing \texttt{atol} leads to less stringent
energy conservation.

\begin{figure}
    \centering
    \includegraphics[width=0.55\textwidth]{../tests/twostream/figure3.pdf}
    \includegraphics[width=0.55\textwidth]{../tests/twostream/figure3b.pdf}
    \caption{Fluctutations in total particle momentum (upper panel) and
    numerical collision frequency (lower panel).}
    \label{fig:twostream_fig3}
\end{figure}

We can also check the evolution of the total particle momentum, analogous to
\cite[Fig. 3]{LAPENTA2017349}. Computation of $P_0 = \sum_i m_i v_{beam,x}$ and
$P_x(t) = \sum_i m_i v_{i,x}$ is straight forward. Setting up an initial state
with zero momentum is quite a bit harder. The particle number is so low, that
generating $N_p$ independent normally distributed numbers will NOT sum to zero
(or close) to it. Hence the generation of pair particles with anti-correlated
thermal velocities described above. Using that we get the results in
Fig.~\ref{fig:twostream_fig3} that compare favourably with the expected
performance of the method.

Computation of the effective numerical collision frequency $\nu(t) =
\frac{\mathrm{d}\,P_x(t)}{\mathrm{d}t}\frac{1}{P_0}$ is also straight forward.
Taking the RMS value of $\nu(t)$ over the entire duration of the simulation
yields $\langle \nu \rangle = 1.2\cdot10^{-3}\omega_{pe}$. Restricting the time
to $20 \omega_{pe}^{-1} < t 125 \omega_{pe}^{-1}$ increases the value slightly
to $\langle \nu \rangle = 1.3\cdot10^{-3}\omega_{pe}$.

More carefull particle initalization by creating pairs of particles with
opposite $v_x$ to ensure $P_x(0) = 0$ would reduce the osciallations of
momentum between particles and field and bring down the numerical
collisionality.

\begin{figure}
    \centering
    \includegraphics[width=0.75\textwidth]{../tests/twostream/energy.png}
    \caption{Energy breakdown of the 1d3v simulation. All of the energy is initially in kinetic energy of the streaming electrons. Some if it is transferred to $E_x$ by the instability. The other fields are transverse and only populated by fluctuation in the thermal velocities $v_y,z$ of the particles. They contain two (initially) to five (at the peak of the instability) order of magnitude less energy. $B_x$ stays uniformly and statically at $B_x = 0$ up to machine precision, as required by the divergence constraint.}
    \label{fig:twostream_energy}
\end{figure}

The last result that we consider is the break down of energy into components in
the 1d3v run, as shown in \ref{fig:twostream_energy}. Thermal fluctuations
drive transverse fields that quickly reach a relatively steady level. These
fields will be analyzed in more detail in \ref{subsec:wavemodes}.

\FloatBarrier

\subsection{Weibel Instabilty}

This is a test of the transverse two-stream instability, which is closely
related to the Weibel instability driven by temperature anisotropy.

Pick a reference density $n_0$ for normalization. Compute the plasma frequency
$\omega_{pe} = \sqrt{4\pi\,n_o\,e^2/m_e}$ and skin depth $d_e = c/\omega_{pe}$.
The simulation domain is only resolved in one spatial dimension and is $L_x =
2\pi\,d_e$ long and resolved by $N_x = 64$ cells. This implies $\Delta{x} =
L_x/ N_x = \pi/32 d_e \approx .098 d_e$, or in terms of Debye length $\Delta{x}
\approx 9.8 \lambda_D$. This resolution is borderline, but feasible with
regular PiC codes.  We use a timestep of $\Delta{t} = 1/8 \omega_{pe}^{-1}$ and
follow the system for $T = 125 \omega_{pe}^{-1}$, which implies we perform $N_t
= 1000$ time steps. The CFL condition number is given by $\mathcal{C} = c
\Delta{t}/\Delta{x} = 1/8 c \omega_{pe}^{-1} / (\pi/32 c \omega_{pe}^{-1}) =
4/\pi \approx 1.27$. This is the same CFL condition number before, but this
time in an electromagnetic instability. Expicit PiC codes would have a hard
time with this.  Two electron populations of equal density $n_0/2$ are
initialized with beam speeds $v_{beam,y} = \pm 0.8 c$ and thermal spread
$v_{th,e} = 0.01 c$. Velocity of particle \texttt{i} is given by
\begin{align*}
    v_x[i] &= v_{th,e} * \mathcal{N}(0,1) \\
    v_y[i] & = -1^i * v_{\mathrm{beam},y} + v_{th,e} * \mathcal{N}(0,1) \\
    v_z[i] &= v_{th,e} * \mathcal{N}(0,1)
\end{align*}
We use 154 particles per cell (following \cite{LAPENTA2017349}, even though 156
ppc would get us closer to $10^4$ particles in total).  The growth rate for
this instability is given (in the fluid limit, see e.g.
\url{https://en.wikipedia.org/wiki/Weibel_instability}) by
\begin{equation*}
    \frac{\gamma}{\omega_{pe}} =  \frac{v_0}{c} \sqrt{\frac{k^2 d_e^2}{1+k^2 d_2^2}}
\end{equation*}
For Fourier mode $m$ in a domain of length $L_x = 2\pi d_e$ that has $k_m =
\frac{2\pi\,m}{L_x} = \frac{2\pi\,m}{2\pi\, d_e} = \frac{m}{d_e}$ we expect a
growth rate of
\begin{equation*}
    \frac{\gamma}{\omega_{pe}} =  \frac{v_0}{c} \sqrt{\frac{m^2}{1+m^2}}
\end{equation*}
Energy growth of the different $m$ Fourier modes of $B_z$ are shown in
\ref{fig:weibel_fig4} and match expected growth rates rather well. But for some
reason $m=2$ is the most unstable mode and even seeding $m=3$ doesn't help.
% I have tried perturbation of $v_y$, perturbations of $v_x$, perturbations in
% the initial $B_z$ and perturbation of initial particle position $x_p$, all of
% them following $\sin\left(2\pi\,x[i]/L_x\right)$. None of them makes $m=3$
% the most unstable mode, in all cases $m=2$ is the saturating mode.

\begin{figure}
    \centering
    \includegraphics[width=0.55\textwidth]{../tests/weibel/figure1.png}
    \caption{Compare with \cite[figure 4]{LAPENTA2017349}. For us $m=2$ dominates the dynamics, not $m=3$.}
    \label{fig:weibel_fig4}
\end{figure}

However, conservation of total energy is great as displayed in Fig.~\ref{fig:weibel_fig5}.

\begin{figure}
    \centering
    \includegraphics[width=0.55\textwidth]{../tests/weibel/figure2.pdf}
    \caption{Energy is conserved up to machine precission.}
    \label{fig:weibel_fig5}
\end{figure}

\FloatBarrier

\subsection{Finite-Grid-Instability}

This is repeating the tests in \cite[Section 5]{LAPENTA2017349}. The paper
states the domain length as $L = 2\pi\,\Xi$ which I take to mean $L_x =
2\pi\,\Xi\,d_e$. This domain length is devided into $N_x = 64$ cells. The cell
size is therefore $\Delta{x} = \frac{L_x}{N_x} = \frac{\pi\,\Xi}{32}d_e$. Given
the thermal speed $v_{th} = 0.01 c$ we see that the Debye length is $\lambda_D
= \frac{v_th}{\omega_{pe}} = \frac{1}{100}\frac{c}{\omega_{pe}} = \frac{1}{100}
d_e$. So if we calculate the cell sizes as a multiple of the Debye length we
find
\begin{align}
    \frac{\Delta{x}}{\lambda_D} &= \frac{L_x\,\omega_{pe}}{N_x\, v_{th}}\\
        &= \frac{\omega_{pe}\,2\pi\,\Xi\,d_e}{64\, v_{th}} \label{eqn:sec5_paper_res} \\
        &= \frac{200\pi\,\Xi}{64} \\
        &= \frac{25\pi\,\Xi}{8} \\
        &\approx 9.81 \Xi
\end{align}
Note that \cite{LAPENTA2017349} has an extra factor $L$ in
Eq.~\eqref{eqn:sec5_paper_res}, probably a copy-and-paste mistake. Additionally
$d_e$ is ommited. The paper then states that values of $\Xi \in [1,10^{15}]$
are considered, making $\Delta{x}/\lambda_D \in [10,10^{16}]$ which seems close
enough to what we get.

For the timestep the paper states $\omega_{pe}\Delta{t} = .125\Xi$ which
obviously implies $\Delta{t} = \frac{1}{8}\Xi\omega_{pe}^{-1}$. If I use that
to compute the CFL number
\begin{align}
    \mathcal{C} = \frac{c\,\Delta{t}}{\Delta{x}} &= \frac{c\, \frac{1}{8}\Xi\omega_{pe}^{-1}}{\frac{\pi\,\Xi}{32}d_e} \\
        & = \frac{4}{\pi} \\
        & \approx 1.27
\end{align}
I get a value that is much larger than the quoted ``so that the CFL number is
constant in all runs and equal to 0.0127''.  So potentially he is defining the
CFL number with respect to the thermal speed (if we treat is as an
electrostatic problem, there is no lightwaves). That would get us
\begin{align}
    \mathcal{C'} = \frac{v_{th}\,\Delta{t}}{\Delta{x}} &= \frac{\frac{1}{100}c\, \frac{1}{8}\Xi\omega_{pe}^{-1}}{\frac{\pi\,\Xi}{32}d_e} \\
        & = \frac{4}{100\pi} \\
        & \approx 0.0127
\end{align}
which would match the quoted text, but the paper is not clear in the definition
of the CFL number and there is going to be a few percent of particles that go
faster than $v_{th}$.

The duration of the simulation runs is not reported, so I am picking the number
of time steps to be $N_t = 1000$, which makes $T = 125\Xi\omega_{pe}^{-1}$.

\begin{figure}
    \centering
    \includegraphics[width=0.55\textwidth]{../tests/finite_grid/figure1_1d3v.pdf}
    \caption{Performance for simulations using $\omega_{pe} \Delta{t} = 0.0125$ and various cell sizes $\Delta{x}/\lambda_D$.}
    \label{fig:fgi_fig7}
\end{figure}

\FloatBarrier

\subsection{Wave Modes}
\label{subsec:wavemodes}
These tests follow the setup described in \cite{kilian_2017}, with one exception: The resolved direction is $x$ instead of $z$.

\subsubsection{Electromagnetic Mode}

First we test the dispersion relation of (relatively) high-frequency modes in
an unmagnetized plasma. The first one is the electromagnetic mode, analogous to
\cite[Figure 3]{kilian_2017}. This is shown in Fig.~\ref{fig:waves_0B_B1}. We
see that at large k the numerical dispersion is even large than in an explicit
code with a regular Yee solver. This is not unexpected for an algorithm that
colocated E fields at vertices and B fields at cell centers. But high $k$ is
not the focus of the algorithm. This poor numerical dispersion would likely
lead to problems with relativistic particles through the production of NCI, but
the code is currently non-relativistic anyway.

\begin{figure}
    \centering
    \includegraphics[width=0.55\textwidth]{../tests/waves/run0_disp_E1_x.pdf}
    \caption{Electromagnetic mode at large $k$ and $\omega$.}
    \label{fig:waves_0B_B1}
\end{figure}

If we zoom in to lower $k$ and $\omega$ (as shown in
Fig.~\ref{fig:waves_0B_E1}) we can identify the plasma frequency $w_{pe}$ at
the correct value. This indicates that the normalization of current deposition,
particle pusher and Maxwell solver is consistent. This figure is analogous to
\cite[Figure 2]{kilian_2017}

\begin{figure}
    \centering
    \includegraphics[width=0.55\textwidth]{../tests/waves/run0_disp_E2_x.pdf}
    \caption{Electromagnetic mode at smaller $k$ and $\omega$.}
    \label{fig:waves_0B_E1}
\end{figure}

Electrostatic modes such as the Langmuir mode show the same plasma frequency
and show an evolution with $k$ that shows that the particle thermal speed is
correct in the code. This can be seen in Fig.~\ref{fig:waves_0B_E0} which is
analogous to \cite[Figure 8]{kilian_2017}.

\begin{figure}
    \centering
    \includegraphics[width=0.55\textwidth]{../tests/waves/run1_disp_E0_x.pdf}
    \caption{Electrostatic mode at smaller $k$ and $\omega$.}
    \label{fig:waves_0B_E0}
\end{figure}

\FloatBarrier

\subsubsection{High-frequency L- and R- mode}

Adding a magnetic field along $x$ such that $\omega_{pe}/\Omega_{ce} = 2$ the
electromagnetic mode splits into a left and right handed mode. This is shown in
Fig.~\ref{fig:waves_Bx_E1}, the equivalent of \cite[Figure 5]{kilian_2017} and
Fig.~\ref{fig:waves_Bx_Er}, the equivalent of \cite[Figure 6]{kilian_2017}.
This test is important because it shows that the signs in $\alpha$ is correct
and we are not accidentally using the transpose $\alpha^{T}$.

\begin{figure}
    \centering
    \includegraphics[width=0.55\textwidth]{../tests/waves/run2_disp_El_x.pdf}
    \caption{Left-hand circular polarized electromagnetic modes.}
    \label{fig:waves_Bx_E1}
\end{figure}

\begin{figure}
    \centering
    \includegraphics[width=0.55\textwidth]{../tests/waves/run2_disp_Er_x.pdf}
    \caption{Right-hand circular polarized electromagnetic modes.}
    \label{fig:waves_Bx_Er}
\end{figure}

\FloatBarrier

\subsubsection{Extraordinary mode}

\begin{figure}
    \centering
    \includegraphics[width=0.55\textwidth]{../tests/waves/run3_disp_E2_x.pdf}
    \caption{Extraordinary electromagnetic mode at smaller $k$ and $\omega$.}
    \label{fig:waves_By_E1}
\end{figure}

Rotating the magnetic field to point along $B_y$ allows us to benchmark the
extraordinary (X) mode. Fig.~\ref{fig:waves_By_E1} is the equivalent of
\cite[Figure 7]{kilian_2017}.

\FloatBarrier

\subsubsection{Electron Bernstein modes}

\begin{figure}
    \centering
    \includegraphics[width=0.55\textwidth]{../tests/waves/run3_disp_E0_x.pdf}
    \caption{Electron Bernstein modes.}
    \label{fig:waves_By_E0}
\end{figure}

Analyzing the longitudinal field in the same test clearly shows the electron
Bernstein modes. These are shown in Fig.~\ref{fig:waves_By_E0} is the
equivalent of \cite[Figure 11]{kilian_2017}.

\FloatBarrier

\subsubsection{Low-frequency R mode}

In this test we start testing low-frequency modes that are expensive to
simulate in explicit particle-in-cell codes. The domain length here is $L_x =
8192$cm = 273$d_e$, but using only $N_x = 512$ cells instead of 8192 cells as
in the reference paper. The simulation is run for $T = 966 \omega_{pe}^{-1}$
but using only $N_t = 2560$. This makes this test acutally on of the cheaper
steps in the wave modes validation test.

\begin{figure}
    \centering
    \includegraphics[width=0.55\textwidth]{../tests/waves/run4_disp_Er_x.pdf}
    \caption{Right-hand circular polarized electromagnetic modes at small $k$ and $\omega$. This in particular includes Whistler waves.}
    \label{fig:waves_Bx_Er_lf}
\end{figure}

\FloatBarrier

\subsubsection{Low-frequency L mode}

This test is using a larger domain of $L_x = 546\,d_e$ and using $N_x = 3415$
(much less than the 16384 cells in the reference paper). The simulation covers
$T = 4000 \omega_{pe}^{-1}$ but needs only $N_t = 3534$. To have a nice
low-frequency mode that is left hand circular, this test requires the precence
of a proton species. We use a reduced mass ratio $m_p / m_e = 18.36$ here,
following the reference paper, but with this code much larger mass ratios would
be feasible too.

\begin{figure}
    \centering
    \includegraphics[width=0.55\textwidth]{../tests/waves/run5_disp_El_x.pdf}
    \caption{Left-hand circular polarized electromagnetic modes at small $k$ and $\omega$.}
    \label{fig:waves_Bx_Er_vlf}
\end{figure}

\FloatBarrier

\bibliography{main}% Produces the bibliography via BibTeX.

\end{document}
%

